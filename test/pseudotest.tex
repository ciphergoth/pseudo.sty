\documentclass[a4paper]{article}

\def\id{my previous definition}

\usepackage{pseudo}

\title{Tests for the \textsf{pseudo} package}
\author{Magnus Lie Hetland}

\begin{document}
\maketitle

\noindent
This document is an attempt to use the features of the \textsf{pseudo} package
without any other dependencies (as opposed to the \textsf{pseudo}
documentation), e.g., to determine whether it's usable with an older \TeX\
distribution.

\section*{Fonts and styling}

The previous definition of \verb|\id| is: \id\\
\let\id\pseudoid

\bigskip
\noindent
\begin{tabular}{@{}ll@{}}
\verb|\cmd|       & \verb|\pseudocmd|\\
\kw{while}        & \pseudokw{while}\\
\cn{false}        & \pseudocn{false}\\
\id{rank}         & \pseudoid{rank}\\
\st{Hello!}       & \pseudost{Hello!}\\
\pr{Euclid}(a b)  & \pseudopr{Euclid}(a b)\\
\fn{length}(A)    & \pseudofn{length}(A)\\
\fn{length}[A]    & \pseudofn{length}[A]\\
\ct{Important!}   & \pseudoct{Important!}\\
\end{tabular}

{\pseudoset{
    kwfont=KW:~,
    cnfont=CN:~,
    idfont=ID:~,
    stfont=ST:~,
    st-left=[,
    st-right=],
    prfont=PR:~,
    fnfont=FN:~,
    ctfont=CT:~,
    ct-left=[,
    ct-right=],
}
\bigskip
\noindent
\begin{tabular}{@{}ll@{}}
\verb|\cmd|       & \verb|\pseudocmd|\\
\kw{while}        & \pseudokw{while}\\
\cn{false}        & \pseudocn{false}\\
\id{rank}         & \pseudoid{rank}\\
\st{Hello!}       & \pseudost{Hello!}\\
\pr{Euclid}(a b)  & \pseudopr{Euclid}(a b)\\
\fn{length}(A)    & \pseudofn{length}(A)\\
\fn{length}[A]    & \pseudofn{length}[A]\\
\ct{Important!}   & \pseudoct{Important!}\\
\end{tabular}}

% The one on the line is ignored:
\begin{pseudo}[hpad, dim-color=red, kw][dim-color=blue, dim]
    foo \\[hl]
    bar
\end{pseudo}

\noindent
Should be \verb|\fn{test}[A]| followed by \verb|(B)| in normalfont:
\fn{test}[A](B)

\bigskip
\noindent
Should be \verb|\fn{test}(A)| followed by \verb|[B]| in normalfont:
\fn{test}(A)[B]

\section*{Declarations and definitions}

\DeclarePseudoComment \Imp {Important!}
\DeclarePseudoConstant \False {false}
\DeclarePseudoFunction \Ln {length}
\DeclarePseudoIdentifier \Rank {rank}
\DeclarePseudoKeyword \While {while}
\DeclarePseudoNormal \Error {halt with an error message}
\DeclarePseudoProcedure \Euclid {Euclid}
\DeclarePseudoString \Hello {Hello!}

\begin{pseudo*}
$x = y$ \qquad \Imp \\
\False \\
\Ln(A) or \Ln[A] \\
\Rank \\
\While \\
\Error \\
\Euclid(a, b) \\
\Hello
\end{pseudo*}

\pseudodefinestyle{mystyle}{
    font = \Large
}

\begin{pseudo*}[mystyle]
Hello, world!
\end{pseudo*}

\section*{Notation}

$A[1\..n] \== B[2\..n+1]$ \\
$A[1\dts n] \eqs B[2\dts n+1]$ \\

\noindent
$\id{foo-bar:baz}$

\noindent
{\pseudoset{eqs-pad=10mu, eqs-scale=2, eqs-sep=5mu}$x\==y$}

\begin{pseudo*}[kw]
keyword \tn{normal} keyword \nf normal
\end{pseudo*}

\section*{Numbering and indentation}

\begin{pseudo}[start=2, label=(\Roman*), label-align=l, ref=\Roman*]
    one \\
    two \label{linetwo} \\*
? & three \\
\end{pseudo}

\begin{pseudo}*
\hd{Header}(\id{args}) \\
first line
\end{pseudo}

\noindent
Reference to line~\ref{linetwo}.

\begin{pseudo}[hpad=1cm, hsep=.1cm, indent-length=2cm, indent-level=1]
A \\+
B \\++
C \\-
D \\---
E
\end{pseudo}

\begin{pseudo*}[indent-text={otherwise\ }]
if this \\+
    then that \\-
otherwise something else
\end{pseudo*}

\section*{Layout}

Here's a compact piece of pseudocode:
\begin{pseudo*}[kw, compact]
print \st{42}
\end{pseudo*}

\noindent
Automatically compact:
\fbox{\begin{pseudo*}
In a box.
\end{pseudo*}}

\begin{pseudo}[left-margin=\parindent]
indented
\end{pseudo}

\begin{pseudo}[line-height=2]
lines \\
far \\
apart
\end{pseudo}

\begin{pseudo}[topsep=1cm, partopsep=1cm]
custom topsep and partopsep
\end{pseudo}

\noindent
Adjustments based on depth:

\bigskip

{

\parskip0pt % Default
\topsep0pt
\partopsep0pt
\noindent
\begin{minipage}[t]{2em}
.

\begin{pseudo*}[hl]
(x) \\
x \\
x \\
(x)
\end{pseudo*}%
\end{minipage}%
%
%
%
\begin{minipage}[t]{2em}
.

\begin{pseudo*}[hl]
x \\
x \\
x \\
x
\end{pseudo*}%
\end{minipage}%
%
%
%
\begin{minipage}[t]{2em}%
.

\begin{tabular}[t]{@{}l@{}}
x \\
x \\
x \\
x
\end{tabular}%
\end{minipage}
%
%
%
\begin{minipage}[t]{2em}%
.

x

\begin{tabular}[t]{@{}l@{}}
x \\
x
\end{tabular}%

\prevdepth=.3\baselineskip

x
\end{minipage}%
%
%
%
\begin{minipage}[t]{2em}
.

x

\begin{pseudo*}[hl]
x \\
x
\end{pseudo*}%

\noindent
x
\end{minipage}%
%
%
%
\begin{minipage}[t]{2em}
.

g

\begin{pseudo*}[hl]
x \\
x
\end{pseudo*}%

\noindent
x
\end{minipage}%
%
%
%
\begin{minipage}[t]{2em}
.

x

\begin{pseudo*}[hl]
x \\
x
\end{pseudo*}%

\noindent
A
\end{minipage}%
%
%
%
\begin{minipage}[t]{2em}
.

(x)

\begin{pseudo*}[hl]
x \\
x
\end{pseudo*}%

\noindent
(x)
\end{minipage}%
%
%
%
\begin{minipage}[t]{2em}
.

\strut x

\begin{pseudo*}[hl]
x \\
x
\end{pseudo*}%

\noindent
\strut x
\end{minipage}%
}

\section*{Overlays etc.}

Ignored outside \textsf{beamer}.

\begin{pseudo}[pause, kwfont<3>=\nf, unknown<3>=42]
    foo \\<2>
    % bar \\ <3> % Dosn't work in older xparse versions
    baz
\end{pseudo}

% Things that could also be tested:
% begin-tabular
% bol
% bol-append
% bol-prepend
% eol
% eol-append
% eol-prepend
% preamble
% prefix
% pseudoeq
% pseudoslash
% setup
% setup-append
% setup-prepend
% starred

\end{document}
